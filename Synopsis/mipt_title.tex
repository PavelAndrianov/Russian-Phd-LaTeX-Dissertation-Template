\thispagestyle{empty}

{
\small
\begin{center}
Федеральное государственное автономное образовательное учреждение \par
высшего образования \par
«Московский физико-технический институт (государственный университет)» \par
кафедра системного программирования
\end{center}

\textbf{Направление подготовки:} 05.13.11 \thesisSpecialtyTitle \par
\textbf{Направленность (профиль) подготовки:} 09.06.01 Информатика и вычислительная техника \par
\textbf{Форма обучения:} очная

\begin{flushright}
\begin{tabular}{ll}
& Допущен к ГИА \\
& Заведующий кафедрой \\
& член-корр. РАН, профессор, д.ф.-м.н.\\
& \underline{\hspace{5em}}/Аветисян А.И. \\
& «\underline{\hspace{2em}}» \underline{\hspace{5em}} 2018 г. \\
\end{tabular}
\end{flushright}
\vspace{0pt plus1fill}
}

\begin{center}
\textbf{
\MakeUppercase{научный доклад}} \par
\textbf{
\MakeUppercase{об основных результатах подготовленной}} \par
\textbf{
\MakeUppercase{научно-квалификационной работы (диссертации)}}
\end{center}

\vspace{0pt plus1fill} %число перед fill = кратность относительно некоторого расстояния fill, кусками которого заполнены пустые места
\begin{center}
\textbf {\thesisTitle}
\end{center}

\vspace{0pt plus2fill}
{
\small
\begin{flushright}
\begin{tabular}{ll}
& \textbf {Аспирант:} \thesisAuthor \\
& \underline{\hspace{12em}} \\
& \textbf {Научный руководитель:} \supervisorFioShort \\
& \supervisorRegaliaShort , ведущий научный сотрудник \\
& \underline{\hspace{12em}} \\
\end{tabular}
\end{flushright}
}

\vspace{0pt plus4fill} %число перед fill = кратность относительно некоторого расстояния fill, кусками которого заполнены пустые места
{\centering\thesisCity~--- \thesisYear\par}
