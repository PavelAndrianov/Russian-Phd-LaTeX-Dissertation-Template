%% Согласно ГОСТ Р 7.0.11-2011:
%% 5.3.3 В заключении диссертации излагают итоги выполненного исследования, рекомендации, перспективы дальнейшей разработки темы.
%% 9.2.3 В заключении автореферата диссертации излагают итоги данного исследования, рекомендации и перспективы дальнейшей разработки темы.

%\begin{enumerate}
%  \item Был разработан общий алгоритм анализа, позволяющий объединить в себе как классический алгоритм CPA, так и алгоритм с раздельным анализом потоков;
%  \item Была доказана теорема, позволяющая гарантировать корректность работы алгоритма;
%  \item На основе общего алгоритма был разработан алгоритм с раздельным анализом потоков;
%  \item Был разработан метод поиска состояний гонки, базирующийся на подходе с раздельным анализом потоков;
%  \item Была доказана корректность предложенного метода поиска состояний гонки, который позволяет гибко балансировать точность и скорость анализа;
%  \item Были проведены эксперименты, которые показали высокую степень гибкости разработанного метода.
%\end{enumerate}

\begin{enumerate}
  \item Был разработан метод поиска состояний гонки на основе раздельного анализа потоков, использующий средства абстракции состояний и переходов для управления точностью и ресурсоемкостью верификации.
  \item Был разработан алгоритм построения окружения потока, который позволяет гибко настраивать уровень абстракции над взаимодействием потоков, и доказана его корректность.
  \item Был разработан новый алгоритм, который является обобщением существующего алгоритма cтатической верификации программ при помощи метода CPA, расширяющий типовой набор CPA-анализаторов средствами верификации многопоточных программ с раздельным анализом потоков, и доказана его корректность.
  %\item Были проведены запуски инструмента на модулях ядра операционной системы Linux, а также на двух операционных системах реального времени;
\end{enumerate}

Эксперименты показали преимущества подхода на больших верификационных задачах перед существующими техниками статической верификации.
Небольшие задачи со сложным взаимодействием потоков лучше решаются другими инструментами, так как предложенный подход абстрагируется от такого взаимодействия, что приводит к потере точности, существенной для небольших искусственных задач.
Однако разработанный подход также не пропускает ошибок (при некоторых предположениях) и может быть развит в будущем.

Таким образом, можно заключить, что основные требования к новому инструменту были выполнены, так как он успешно применяется к различным программным системам, в том числе, к драйверам ОС Linux и ядрам ОС реального времени.
Также как и в классических методах статической верификации, может быть получена гарантия отсутствия ошибок при выполнении указанных предположений: требований на операторы CPA и условий корректного разбиения на непересекающиеся регионы BnB модели.
Предложенная общая теория позволяет описывать достаточно сложные варианты анализа, в том числе и те, которые не включаются в понятие анализа с раздельным рассмотрением потоков.
Тем не менее, эта теория не исчерпывает всех возможных подходов, и для эффективного описания масштабируемого подхода с частичным вычислением чередований потребуется ее расширение. Однако, эта тема выходит за рамки данной работы.
Одним из возможных направлений развития подхода является добавление взаимодействия потоков, возможно, не в полном объеме, чтобы сохранить масштабируемость.
Это может быть отдельный анализ CPA, который будет динамически настраивать свою точность с помощью алгоритма CEGAR, обеспечивая некоторый промежуточный вариант между подходом с раздельным анализом потоков и перебором чередований.
Другим возможным улучшением инструмента может стать интеграция его с другим подходом. 
Например, комбинация быстрого подхода с раздельным анализом потоков в качестве первого этапа, а затем классический тяжеловесный анализ – на втором этапе. Это может быть реализовано в соответствии с идеей кооперативной верификации.
Еще одним возможным направлением развития подхода является построение реального пути с чередованием потоков на основе пути с примененными эффектами окружения. Этот путь был бы полезен для исследования и уточнения абстракции. 
