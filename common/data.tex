%%% Основные сведения %%%
\newcommand{\thesisAuthorLastName}{Андрианов}
\newcommand{\thesisAuthorOtherNames}{Павел Сергеевич}
\newcommand{\thesisAuthorInitials}{П.\,С.}
\newcommand{\thesisAuthor}             % Диссертация, ФИО автора
{%
    \texorpdfstring{% \texorpdfstring takes two arguments and uses the first for (La)TeX and the second for pdf
        \thesisAuthorLastName~\thesisAuthorOtherNames% так будет отображаться на титульном листе или в тексте, где будет использоваться переменная
    }{%
        \thesisAuthorLastName, \thesisAuthorOtherNames% эта запись для свойств pdf-файла. В таком виде, если pdf будет обработан программами для сбора библиографических сведений, будет правильно представлена фамилия.
    }
}
\newcommand{\thesisAuthorShort}        % Диссертация, ФИО автора инициалами
{\thesisAuthorInitials~\thesisAuthorLastName}
%\newcommand{\thesisUdk}                % Диссертация, УДК
%{\todo{xxx.xxx}}
\newcommand{\thesisTitle}              % Диссертация, название
{Анализ корректности синхронизации компонентов ядра операционных систем}
\newcommand{\thesisSpecialtyNumber}    % Диссертация, специальность, номер
{05.13.11}
\newcommand{\thesisSpecialtyTitle}     % Диссертация, специальность, название
{Математическое и программное обеспечение вычислительных машин, комплексов и компьютерных сетей}
\newcommand{\thesisDegree}             % Диссертация, ученая степень
{кандидата физико-математических наук}
\newcommand{\thesisDegreeShort}        % Диссертация, ученая степень, краткая запись
{к.~ф.-м.~н.}
\newcommand{\thesisCity}               % Диссертация, город написания диссертации
{Москва}
\newcommand{\thesisYear}               % Диссертация, год написания диссертации
{2020}
\newcommand{\thesisOrganization}       % Диссертация, организация
{Институт системного программирования им. В.П. Иванникова \\ Российской академии наук}
\newcommand{\thesisOrganizationShort}  % Диссертация, краткое название организации для доклада
{ИСП РАН}

\newcommand{\supervisorFio}            % Научный руководитель, ФИО
{Хорошилов Алексей Владимирович}
\newcommand{\supervisorRegalia}        % Научный руководитель, регалии
{кандидат физико-математических наук}
\newcommand{\supervisorFioShort}       % Научный руководитель, ФИО
{Хорошилов~А.~В.}
\newcommand{\supervisorRegaliaShort}   % Научный руководитель, регалии
{к.~ф.-м.~н.}


\newcommand{\opponentOneFio}           % Оппонент 1, ФИО
{Галатенко Владимир Антонович}
\newcommand{\opponentOneRegalia}       % Оппонент 1, регалии
{доктор физ.-мат. наук, старший научный сотрудник}
\newcommand{\opponentOneJobPlace}      % Оппонент 1, место работы
{Федеральное государственное учреждение ``Федеральный научный центр Научно-исследовательский институт системных исследований РАН''}
\newcommand{\opponentOneJobPost}       % Оппонент 1, должность
{заведующий сектором}

\newcommand{\opponentTwoFio}           % Оппонент 2, ФИО
{Романенко Сергей Анатольевич}
\newcommand{\opponentTwoRegalia}       % Оппонент 2, регалии
{кандидат физико-математических наук}
\newcommand{\opponentTwoJobPlace}      % Оппонент 2, место работы
{Федеральное государственное учреждение ``Федеральный исследовательский центр Институт прикладной математики им. М.В. Келдыша РАН''}
\newcommand{\opponentTwoJobPost}       % Оппонент 2, должность
{ведущий научный сотрудник}

\newcommand{\leadingOrganizationTitle} % Ведущая организация, дополнительные строки
{Федеральное государственное учреждение Федеральный исследовательский центр Информатика и управление Российской академии наук}

\newcommand{\defenseDate}              % Защита, дата
{18 июня 2020~г.~в~15 часов}
\newcommand{\defenseCouncilNumber}     % Защита, номер диссертационного совета
{Д\,002.087.01}
\newcommand{\defenseCouncilTitle}      % Защита, учреждение диссертационного совета
{Федеральном государственном бюджетном учреждении науки Институте системного программирования им. В.П.Иванникова РАН}
\newcommand{\defenseCouncilAddress}    % Защита, адрес учреждение диссертационного совета
{109004, г. Москва,ул. А. Солженицына, дом 25}
\newcommand{\defenseCouncilPhone}      % Телефон для справок
{\todo{+7~(0000)~00-00-00}}

\newcommand{\defenseSecretaryFio}      % Секретарь диссертационного совета, ФИО
{Зеленов С. В.}
\newcommand{\defenseSecretaryRegalia}  % Секретарь диссертационного совета, регалии
{кандидат физ.-мат. наук}            % Для сокращений есть ГОСТы, например: ГОСТ Р 7.0.12-2011 + http://base.garant.ru/179724/#block_30000

\newcommand{\synopsisLibrary}          % Автореферат, название библиотеки
{\todo{Название библиотеки}}
\newcommand{\synopsisDate}             % Автореферат, дата рассылки
{\underline{\hspace{2cm}} 2020 года}

% To avoid conflict with beamer class use \providecommand
\providecommand{\keywords}%            % Ключевые слова для метаданных PDF диссертации и автореферата
{}
