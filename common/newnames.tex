% Новые переменные, которые могут использоваться во всём проекте
% ГОСТ 7.0.11-2011
% 9.2 Оформление текста автореферата диссертации
% 9.2.1 Общая характеристика работы включает в себя следующие основные структурные
% элементы:
% актуальность темы исследования;
\newcommand{\actualityTXT}{Актуальность темы исследования.}
% степень ее разработанности;
\newcommand{\progressTXT}{Степень разработанности темы.}
% цели и задачи;
\newcommand{\aimandtasksTXT}{Цели и задачи.}
\newcommand{\aimTXT}{Целью}
\newcommand{\tasksTXT}{задачи}
% научную новизну;
\newcommand{\noveltyTXT}{Научная новизна:}
% теоретическую и практическую значимость работы;
\newcommand{\influenceTXT}{Теоретическая и практическая значимость работы.}
% или чаще используют просто
%\newcommand{\influenceTXT}{Практическая значимость}
% методологию и методы исследования;
\newcommand{\methodsTXT}{Mетодология и методы исследования.}
% положения, выносимые на защиту;
\newcommand{\defpositionsTXT}{Положения, выносимые на~защиту:}
% степень достоверности и апробацию результатов.
\newcommand{\reliabilityTXT}{Достоверность}
\newcommand{\probationTXT}{Апробация результатов.}

\newcommand{\contributionTXT}{Личный вклад автора.}
\newcommand{\publicationsTXT}{Публикации.}


\newcommand{\authorbibtitle}{Публикации автора по теме диссертации}
\newcommand{\vakbibtitle}{В изданиях из списка ВАК РФ}
\newcommand{\notvakbibtitle}{В прочих изданиях}
\newcommand{\confbibtitle}{В сборниках трудов конференций}
\newcommand{\fullbibtitle}{Список литературы} % (ГОСТ Р 7.0.11-2011, 4)

\newcommand{\conc}[1]{[\![ #1 ]\!]}
\newcommand{\tconc}[2]{[\![ #1 ]\!]_{#2}}
\newcommand{\pconc}[1]{\lfloor \! \lfloor #1 \rfloor \! \rfloor}
\newcommand{\econc}[1]{\lvert \! \lvert #1 \rvert \! \rvert}

\newcommand{\tcarrow}{\longrightarrow}
\newcommand{\tatarrow}{\leadsto}
\newcommand{\taearrow}{\Rightarrow}
\newcommand{\tc}[1]{\stackrel{#1}{\longrightarrow}}
\newcommand{\tat}[1]{\stackrel{#1}{\leadsto}}
\newcommand{\tae}[1]{\stackrel{#1}{\Rightarrow}}

\newcommand{\alias}{\stackrel{\rm a}{=}}
\newcommand{\ite}[3]{ite.\enskip{#1}.\enskip{#2}.\enskip{#3}}

\newcommand{\arc}[1]{$\stackrel{\rm {#1}}{\longrightarrow}$}
\newcommand{\pname}[1]{\langle{#1}\rangle}

\newcommand{\reduce}[1]{\stackrel{\rm {#1}}{\longrightarrow}}
\newcommand{\corresp}{\leftrightarrow}

\newcommand{\ipar}[1]{\par\hspace{#1} }
\newcommand{\iepar}[2]{\par{#2}\hspace{#1} }
\newcommand{\abs}[2]{{#1}:{\{#2\}}}
\newcommand{\trans}[5]{\abs{#1}{#2} \arc{#3} \abs{#4}{#5}}


\renewcommand{\theenumi}{\arabic{enumi}}
\renewcommand{\theenumii}{\arabic{enumii}}
\renewcommand{\theenumiii}{\arabic{enumiii}}

\renewcommand{\labelenumi}{\theenumi.}
\renewcommand{\labelenumii}{\theenumi.\theenumii.}
\renewcommand{\labelenumiii}{\theenumi.\theenumii.\theenumiii.}

\newcommand*{\qdot}{%
  \mathclose{}%
  \nonscript\mskip.5\thinmuskip
  \boldsymbol{.}%
  \;%
  \mathopen{}%
}

\newtheorem{thrm}{Теорема}
\newtheorem{stmnt}{Утверждение}
\newtheorem{defn}{Определение}
\newtheorem{exmp}{Пример}

\makeatletter
\renewcommand{\p@enumii}{\theenumi.}
\renewcommand{\p@enumiii}{\theenumi.\theenumii.}
\makeatother

\newcommand{\cpachecker}{CPAchecker}
\newcommand{\cpalockator}{CPALockator}

\newcommand{\mynote}[1]{\textsf{$\clubsuit$ #1 $\clubsuit$ }}